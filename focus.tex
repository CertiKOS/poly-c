\documentclass[nocopyrightspace,preprint]{sigplanconf-pldi15}

\usepackage[utf8]{inputenc} %for utf8 input
\usepackage{amssymb} %for shift symbol
\usepackage{amsmath}
\usepackage{listings} %for code
\usepackage{mathpartir} %for typing rules
\usepackage{microtype} %better micro typing
\usepackage{stmaryrd} %for llbracket
\usepackage{mathabx} % for boxes
\usepackage{graphicx} %to include png images
\usepackage{xcolor} %for colors
\usepackage{url}
\usepackage{enumitem}
\usepackage{array} %for stupid tables

%----------------------------------------------------

\usepackage{prettyref}
\newcommand{\pref}[1]{\prettyref{#1}}
\newcommand{\Pref}[1]{\prettyref{#1} \vpageref[]{#1}}
\newcommand{\ppref}[1]{\vpageref[]{#1}}
\newrefformat{fig}{Figure~\ref{#1}}
\newrefformat{app}{Appendix~\ref{#1}}
\newrefformat{tab}{Table~\ref{#1}}
\newrefformat{cha}{Challenge~\ref{#1}}
\newrefformat{thm}{Theorem~\ref{#1}}

%----------------------------------------------------

\usepackage{amsthm}
\newtheorem{lemma}{Lemma}
\newtheorem{corollary}{Corollary}
\newtheorem{theorem}{Theorem}
\newtheorem{proposition}{Proposition}

%----------------------------------------------------

\input{definitions}
\begin{document}

\section{Potential}

To be able to track naturally more program statements
we change the typical potential from a linear combination
of (globally non-negative) base potentials to an arbitrary
polynomial expression of program variables. Monomials are
defined by $M$ and potential functions are defined by $P$.
%
\begin{align*}
M &:= 1 ~\vert~ x ~\vert~ \max(P_1, P_2) ~\vert~ M_1 \cdot M_2 \\
P &:= k \cdot M ~\vert~ P_1 + P_2
\end{align*}
%
Where $k \in \mathbb Z$ and $x$ is any program variable.
The grammar is taken modulo associativity and commutativity
of the multiplication and addition symbols. (We could also
consider assiciativity and commutativity of the $\max$ symbol.)

As a consequence of the previous definition we have that
substituting any variable in a potential function by a
polynomial of program variables will yield another potential
function.

\section{Focus Functions}

Focus functions are potentials of special interest.
To be useful, they must be expressions that are non-negative
at some critical program points.  Here are some examples
of focus functions:
%
\begin{align*}
m_{xy}^1 &:= \max(x,y) - x \\
m_{xy}^2 &:= \max(x,y) - y \\
m_{xyz}^1 &:= \max(x, y) + z - \max(x,y+z) \\
i_{xy} &:= \max(0, y - x) \\
b(3, m_{xyz}^{1}) &:= \binom{\max(x,y) + z}{3} - \binom{\max(x, y+z)}{3}.
\end{align*}
%
Both $m_{xy}^1$ and $m_{xy}^2$ are unconditionally non-negative.
But $m_{xyz}^2$ can sometimes be negative, a sufficient condition
for it to be non-negative is $z \ge 0$.

Focus functions are used to enforce constraints between
general potential functions.  In a program state $\sigma$,
we can constrain $P \ge P'$ using the following rule:

$$
\Rule{Const}
{
  \forall i, \sigma \models F_i(\sigma) \ge 0 \\
  \forall i, k_i \ge k'_i \\\\
  P = L + \sum_i k_i \cdot F_i \\
  P' = L + \sum_i k'_i \cdot F_i
}
{P(\sigma) \ge P'(\sigma)}
$$

To prove non-negativity of focus functions at a given program
point we proceed in two steps. First, we compute an invariant
respected by this point using abstract interpretation.  Second,
we reduce the complexity of the focus function using sufficient
conditions (in the example above $b(3, m_{xyz}^1) \ge 0 \impliedby
m_{xyz}^1 \ge 0 \impliedby z \ge 0$), and then resolve the base
case using the abstract state computed in the first step.  Note
that to make the second step of this method work it is critical
to have focus functions given in a \emph{structured language},
unlike general potential.

\section{Automation}

\paragraph{Clarification for Jan.} The automation will produce
judgements of the form $\htriple{P}{c}{P'}$ where $P$ and $P'$
are given by the grammar in the first section.  The coefficients
of \emph{toplevel} monomials in $P$ and $P'$ will be LP variables.
In the {\sc Const} rule of the second section $k_i$, $k'_i$ and
the coefficients of $L$ will also be LP variables.

\paragraph{Challenges.}
Several challenges are raised by the automation:

\begin{itemize}
\item Have potential in ``canonical'' form.  If we do not do
  so we might miss trivial facts, like $xy = yx$, and split
  the potential of equal things.
\item Focus functions must be tied to a syntactical scope.
  Since my scopes are currently per-function, it means that focus
  functions must be given per-function.
\item What initial set of monomials do we consider?  I thought
  this could be simply collecting the monomials resulting from the
  development of the current focus functions.
\item As Jan noted, potential functions form a ``language'' that
  might have some consistency requirements to make the automation
  work correctly.  I am not too concerned about this, since focus
  functions encode very local knowledge.
\end{itemize}

My opinion is that most of the issues tied to automation (unless
the scope one) should not affect the theoretical presentation.

\section{Examples}

\paragraph{Linear example.}  The example program below does not
work with the automation presented at PLDI'15, because one ad-hoc
rule for potential transfer between $|[a,b]|$ and $|[c,d]|$ when
$b-a \ge d-c$ is not present in the {\sc Relax} rule.

\begin{lstlisting}[mathescape]
x = 0;
z = 0;
${\color{blue}\{ \max(0, N-x) + z \}}$
while x < N do
  ${\color{blue}\{ \max(0, N-x) + z \}}$
  x = x + 1;
  ${\color{blue}P_1 := \{ \max(0, N-x+1) +z \}}$
  ${\color{blue}P_2 := \{ \max(0, N-x) + 1 + z \}}$
  if x < n then
    ${\color{blue}P_3 := \{ \max(0,N-x) + 1 + z \}}$
    ${\color{blue}P_4 := \{ \max(0,N-n) + 1 + z \}}$
    x = n;
    ${\color{blue}\{ \max(0,N-x) + 1 + z \}}$
  end
  z = z + 1;
  ${\color{blue}\{ \max(0, N-x) + z \}}$
end
${\color{blue}\{ z \}}$
\end{lstlisting}
%
On the previous example we use the following three focus
functions.  They are both instances of more general patterns
that could be added automatically by a heuristic looking at
loop conditions and operations on variables.
\begin{itemize}
\item $F_1 := \max(0, N-x+1) - (N-x+1)$
\item $F_2 := (N-x) - \max(0, N-x)$
\item $F_3 := \max(0, N-n + (n-x)) - \max(0,N-n)$
\end{itemize}

They can be used to prove that $P_1 \ge P_2$ and $P_3 \ge P_4$.
To prove the first identity we first use abstract interpretation
to remark that $F_1$ and $F_2$ are non-negative, then we
rewrite $P_1$ and $P_2$ as
\begin{align*}
P_1 &:= 1 \cdot F_1 + 1 \cdot F_2 + \max(0, N-x) + 1 + z \\
P_2 &:= 0 \cdot F_1 + 0 \cdot F_2 + \max(0, N-x) + 1 + z.
\end{align*}
Note that the coefficients in front of the two focus functions
decreased and the rest remained the same.  This is a valid
application of the {\sc Const} rule.
The second identity is proved by observing that $F_3$ is
non-negative and using the following rewrites
\begin{align*}
P_3 &:= 1 \cdot F_3 + \max(0, N-n) + 1 + z \\
P_4 &:= 0 \cdot F_3 + \max(0, N-n) + 1 + z.
\end{align*}

Thanks to our general potential functions, the rule used to
handle assignments in the program is very simple, I give it below.
Note how it applies at all assignments in the example program.
$$
\Rule{Assign}
{ }
{\htriple{P'[x \gets E]}{\lstinline{x = E;}}{P'}}
$$

\paragraph{Polynomial example.} Using the binomial basis in
focus functions, we can give tight bounds on polynomial
examples.

\begin{lstlisting}[mathescape]
assume x >= 0;
${\color{blue}\{ \frac{x^2-x}{2} \}}$
z = 0;
${\color{blue}\{ \frac{x^2-x}{2} + z \}}$
while x > 0 do
  ${\color{blue}\{ \frac{x^2-x}{2} + z \}}$
  x = x - 1;
  ${\color{blue}\{ \frac{x^2-x}{2} + x + z \}}$
  z = z + x;
  ${\color{blue}\{ \frac{x^2-x}{2} + z \}}$
end
${\color{blue}\{ P_1 := \frac{x^2+x}{2} + z \}}$
${\color{blue}\{ P_2 := z \}}$
\end{lstlisting}

The only application of the {\sc Const} rule is at the very
end of the program to prove that $P_1 \ge P_2$.  To make it
work, we use the focus function $\binom{x}{2}$ and prove that
it is non-negative using the fact that $x \ge 0$ at that point.

\section{Relation to Abstract Interpretation}

Amortized analysis as we use it has several key differences
with abstract interpretation.  Note first that precise amortized
analysis is only possible using information inferred with abstract
interpretation (e.g.\ to check the sign of focus funtions).
However, two main differences make the two kinds of analysis
orthogonal:

\begin{enumerate}

\item Amortized analysis is \emph{non-local}. The properties
we obtain with AA relate variables at different program points,
while AI gives relations between the value of variables at the
same program point.  It is possible to emulate AA using AI by
duplicating all variables at one program point, however, even
for weakly relational abstract domains, there is at least a
quadratic blowup in the number of tracked objects.  (This is
not the case with AA.)

\item Amortized analysis has powerful \emph{inferrence
capabilities}.  The natural use-case for AI is checking, for
example, we check that array accesses are in-bounds, we check
that integer operations do not overflow.  On the contrary,
the natural use-case of AA is inferrence, we infer upper or
lower-bounds on expressions.  Since automated AA uses LP
solving that has optimization capabilities, we can also make
sure that the the result of the AA inference is optimal in
some sense (e.g.\ least upper bound, greatest lower bound).

\end{enumerate}

This does not mean that the two anlaysis are incompatible.
First, as noted above, AA needs AI to get anything done.  We could
also envision to use AA to improve AI results and create a
virtuous cycle!  In particular, functions are a common abstraction
of programming languages that are naturally handled by AA (we
can relate the result with the arguments) but do not work great
with AI.  Feeding back information retreived with AA into an AI
analysis will improve the overall results (there is already a
framework for doing that in the AI library used by my tool).


\end{document}
